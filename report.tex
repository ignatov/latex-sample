\documentclass[a4paper,12pt]{article}
\usepackage[T1]{fontenc}
\usepackage[utf8]{inputenc}
\usepackage[english,russian]{babel}
\usepackage{lmodern}
\usepackage{filecontents}
\usepackage{hyperref}

\newcommand{\noop}[1]{}

\usepackage[
	maxcitenames=2,
	style=authoryear,
	bibstyle=numeric,
	uniquelist=minyear,
	natbib=true,
	backend=biber
]{biblatex}

\AtBeginBibliography{%
   \DeclareNameAlias{author}{last-first}%
}
\renewcommand*{\revsdnamepunct}{}

\DeclareFieldFormat{labelnumberwidth}{\mkbibbold{#1.}}

\renewcommand*{\nameyeardelim}{\addcomma\space}

\renewcommand*{\finalnamedelim}{%
  \finalandcomma
  \bibstring{,}\space}

\bibliography{mybib.bib}

\begin{document}
Обломочные осадочные породы состоят из частиц луны~\parencite{luna} и минералов, образовавшихся в результате разрушения более древних магматических, метаморфических или осадочных пород~\parencite{art}.
Эти частицы несут в себе информацию о тектоническоий обстановке источников сноса осадочных пород и осадочных процессах. Долгое время исследователи пытались определить тектоническую обстановку источников сноса с помощью структурных особенностеий песчаников. Были также попытки объединения петрографических данных с гео- химическими данными~\parencite{batlas}. 
В последние десятилетия появилось много работ, в которых используются изотопные неодимовые данные для осадочных пород. Изучение неодимовоий изотопии в основном сводилось к вопросам эволюции коры, хотя в некоторых исследованиях решались также и тектонические задачи. Одной из первых работ~\parencite{ivanov} по объединению геохимических и изотопных методов является работа Макленнана.

\printbibliography
\end{document}