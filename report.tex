

\documentclass[a4paper,12pt]{article}
\usepackage[T1]{fontenc}
\usepackage[utf8]{inputenc}
\usepackage[english,russian]{babel}

\usepackage{lmodern}
\usepackage[
	backend=biber,
	autocite=inline, 
	labelalpha=true, 
	sortcites=true,
	sortlocale=ru_RU,
    url=false, 
    doi=true,
    eprint=false,
    isbn=false,
    style=authoryear-ibid,
    bibstyle=numeric
    % dashed=true
]{biblatex}

\DeclareFieldFormat{labelnumberwidth}{\mkbibbold{#1.}}

% \usepackage[citestyle=authoryear,bibstyle=numeric,firstinits=true,backend=bibtex]{biblatex}


% \renewbibmacro{in:}{%
%   \ifentrytype{article}{}{%
%   \printtext{\bibstring{in}\intitlepunct}}}

\usepackage{filecontents}

\usepackage{hyperref}
% \hypersetup{colorlinks=true}

\bibliography{mybib.bib}

\begin{document}
Обломочные осадочные породы состоят из частиц пород и минералов, об- разовавшихся в результате разрушения более древних магматических, мета- морфических или осадочных пород~\parencite{art}.
Эти частицы несут в себе информацию о тектоническоий обстановке источников сноса осадочных пород и осадочных процессах. Долгое время исследователи пытались определить тектоническую обстановку источников сноса с помощью структурных особенностеий песча- ников. Были также попытки объединения петрографических данных с гео- химическими данными~\parencite{batlas}. 
В последние десятилетия появилось много работ, в которых используются изотопные неодимовые данные для осадочных пород. Изучение неодимовоий изотопии в основном сводилось к вопросам эволюции коры, хотя в некоторых исследованиях решались также и тектонические за- дачи. Одноий из первых работ по объединению геохимических и изотопных методов является работа Макленнана~\parencite{fart}.

% \bibliographystyle{apa}


\printbibliography[sorting=nyt]

\end{document}